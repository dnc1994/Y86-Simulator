%This file contains the LaTeX code of my Y86 Simulator report for my ICS course.
%Author: 章凌豪/Zhang Linghao <zlhdnc1994@gmail.com>

\section{Evaluation}

本节将叙述对模拟器的测试和性能评估。

\subsection{Test Cases}

在{\bf tests}目录下收录有我所使用的测试用例,说明如下:

\begin{table}[h]
\begin{tabular}{|l|l|l|l|}
\hline
{\bf \#} & {\bf 测试文件}                          & {\bf 描述}                  & {\bf 预期结果}                \\ \hline
1        & asum.yo                             & 测试样例,包括.long、循环和调用        & \%eax = 0xabcd            \\ \hline
2        & Halt.ys                       & 测试模拟器是否能能正确结束运行           & \begin{tabular}[c]{@{}l@{}}\%eax = 0x1\\ \%ebx = 0x0\end{tabular} \\ \hline
3        & Forward.ys        & 测试模拟器是否能正确Forward后修改的寄存器值 & \%eax = 0x3               \\ \hline
4        & Load\_Use.ys          & 测试模拟器是否能处理Load/Use Hazard & \%eax = 0xd               \\ \hline
5        & Comb\_A.ys             & 测试模拟器是否能处理Combination A   & \%eax = 1                 \\ \hline
6        & List\_Sum.ys            & 对链表中的三个元素进行求和,测试内存取值相关操作  & \%eax = 0xcba             \\ \hline
7        & List\_Sum\_R.ys & 同上,但用递归完成,测试调用和返回等操作      & \%eax = 0xcba             \\ \hline
\end{tabular}
\end{table}

由于测试1包含了大部分情况,所以在模拟器通过测试1并且将结果与给出的{\bf asum.txt}对比发现基本一致后,可以认为逻辑大体上没有问题。

再通过测试2到测试5,可以基本肯定流水线逻辑(Stall / Bubble / Forward)没有问题。

最后的测试6和测试7是为了进一步验证模拟器的正确性,同时为性能分析提供更多例子。

此外还对几类常见的报错进行了简单的测试。由于时间原因,没有进行更多的测试。

\subsection{Performance}

在Chrome 37.0下测试,得到不限速时运行速度可达$15K cycle/sec$。

同时,在几个测试用例中,缓存命中率均在$25\%$左右。