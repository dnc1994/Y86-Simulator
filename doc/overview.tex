%This file contains the LaTeX code of my Y86 Simulator report for my ICS course.
%Author: 章凌豪/Zhang Linghao <zlhdnc1994@gmail.com>

\section{Overview}

\subsection{Development Environment}

\begin{table}[h]
\begin{tabular}{ll}
{\bf 开发语言}  & JavaScript / HTML / CSS   \\
{\bf 浏览器环境} & Chrome / Firefox / Safari \\
 & \\
 & jQuery                    \\
{\bf 第三方库} & Bootstrap                 \\
 & FileSaver.js              \\
 & Chart.js                 
\end{tabular}
\end{table}

\subsection{File Organization}

\noindent
本次Project的主要代码均在{\bf js}目录下,说明如下:

\begin{table}[h]
\begin{tabular}{ll}
{\bf js/constants.js}    & 一些常量的定义       \\
{\bf js/utils.js}        & 一些辅助函数的实现     \\
{\bf js/registers.js}    & 寄存器和流水线寄存器的实现 \\
{\bf js/memory.js}       & 内存的实现         \\
{\bf js/cache.js}        & 缓存的实现         \\
{\bf js/kernel.js}       & ALU和CPU的实现    \\
{\bf js/assembler.js}    & Y86汇编器的实现     \\
{\bf js/disassembler.js} & Y86反汇编器的实现    \\
{\bf js/controller.js}   & 控制函数的实现      
\end{tabular}
\end{table}
