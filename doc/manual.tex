%This file contains the tex code of my Y86 Simulator manual for my ICS course.
%Author: 章凌豪/Zhang Linghao <zlhdnc1994@gmail.com>

\documentclass[12pt]{article}
\usepackage[top=1.15in, bottom=0.75in, left=1.00in, right=1.00in]{geometry}
\linespread{1.3}
\usepackage{ctex}
\usepackage[colorlinks, citecolor=green, linkcolor=blue, menucolor=red, CJKbookmarks=true]{hyperref}
\usepackage{fancyhdr}
\usepackage{extramarks}
\usepackage{titling}
\usepackage{indentfirst}
\usepackage[linesnumbered]{algorithm2e}
\usepackage{graphicx}
\usepackage{array}
\usepackage{bibentry}
\usepackage{natbib}

\iffalse
\AtBeginDocument{
\begin{CJK*}{GBK}{SimSun}
\CJKindent
\sloppy\CJKspace
\CJKtilde
}
\AtEndDocument{\end{CJK*}}
\fi

\begin{document}

\pagestyle{fancy}
\lhead{\textbf{{\thetitle}}}
\rhead{\textbf{\nouppercase{\firstleftmark}}}
\cfoot{\thepage}

\title{\textbf{Y86 Simulator User Manual}}
\author{章凌豪\\13307130225}
\date{\today}
\maketitle

%\tableofcontents
%\clearpage

\vspace{42pt}

\section{Usage}

本Y86流水线模拟器不需要安装,直接用Chrome / Firefox / Safari等浏览器打开{\bf index.html}即可运行,所有用到的JavaScript和CSS文件都是自带的。由于兼容IE浏览器的工作量过大,所以不支持IE。为防止由于预想之外的原因导致运行表现与预期不一致,用户可以配合附带的demo视频使用本手册。

\section{User Interface}

\begin{enumerate}
\item 将一个.yo或.ys文件拖拽至页面的任一位置即可加载,如果加载不成功会在页面上方控制栏左侧显示错误信息。
\item 对于.yo文件,模拟器会尝试对其进行反汇编;对于.ys文件,模拟器会对其进行汇编。如果没有错误发生,可以通过点击控制栏左侧的“xxx.yo loaded.”打开代码显示窗口进行查看,也可以保存反汇编/汇编的结果。
\item 载入文件后,可以通过控制栏上的五个按钮(或通过键盘快捷键进行控制,详见附录)分别进行步退、暂停、自动运行、步进和重置操作。按钮右边的{\bf Frequency}框可以设定自动运行的频率,最高支持1000HZ;{\bf Mem Addr}框中可以输入要监视的内存地址(以0x开头的十六进制地址),对应的值会在界面最左侧的运行状态栏下{\bf valMem}条目中显示。
\item 界面的主体是五列流水线寄存器的状态显示。在最左侧的Fetch下方可以查看模拟器的运行状态,包括CPU状态、周期数、指令数、监视内存值、CPI和缓存命中统计;在Write back的右侧是8个寄存器的数值显示,当数值发生变化时会触发闪烁动画;再右侧是Condition Codes的显示。
\item 界面的最右侧是内存数据的显示,没有被写过的内存显示为0xCC,有两个指针会分别指示\%ebp和\%esp的位置。出于前端性能考虑,只显示前1024位内存。
\item 界面下方可以保存运行结果,即每个周期内流水线寄存器的数值,也可以自定义保存文件名;在程序运行结束后,可以点击{\bf Perf Analysis}来打开性能分析窗口,其中包含对缓存命中和CPI的统计。
\end{enumerate}

\section{Appendix}

下面列出键盘快捷键对应的操作:

\hspace{20pt}

\begin{table}[h]
\begin{tabular}{|l|l|}
\hline
{\bf 按键} & {\bf 操作} \\ \hline
←        & 步退       \\ \hline
→        & 步进       \\ \hline
回车       & 自动运行     \\ \hline
空格       & 暂停       \\ \hline
R        & 重置       \\ \hline
F        & 输入运行频率   \\ \hline
M        & 输入监视内存地址 \\ \hline
\end{tabular}
\end{table}

\end{document}
