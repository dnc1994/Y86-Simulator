\documentclass[12pt]{beamer}

\usetheme{Warsaw}

\usepackage{xeCJK}
\setCJKmainfont{SimSun}

\usepackage{graphicx}

\usepackage{hyperref}
\hypersetup{
	linkcolor=blue,
	urlcolor=blue,
}

\linespread{1.15}

\newCJKfontfamily\zhexue{SimSun}	

\begin{document}

	\title[CSAPP]{Y86 Pipeline Simulator}
	\author{章凌豪}
	\date{\today}

	\frame{\titlepage}
	
	\begin{frame}[allowframebreaks]{Outline}
		\tableofcontents
	\end{frame}

	\section{开发环境}

	\frame{
		\begin{table}[h]
		\begin{tabular}{ll}
		{\bf 开发语言}  & JavaScript / HTML / CSS   \\
		{\bf 浏览器环境} & Chrome / Firefox / Safari \\
		 & \\
		 & jQuery                    \\
		{\bf 第三方库} & Bootstrap                 \\
		 & FileSaver.js              \\
		 & Chart.js                 
		\end{tabular}
		\end{table}
	
	\begin{itemize}[<+-| alert@+>]
		\item 不适合前后端分离
		\item 配合HTML和CSS,数据表现能力出色
		\item 人生苦短,我用脚本语言
	\end{itemize}
	}
	
	\section{实现功能}
	
	\frame{
		\frametitle{基本功能}
		\begin{itemize}[<+-| alert@+>]
		\item 实现了Y86指令集中的所有指令。
		\item 实现了流水线控制逻辑。
		\item 支持载入和解析.yo文件,并能将每个周期内流水线寄存器的数值作为输出保存到文件。
		\end{itemize}
	}
	
	\frame{
		\frametitle{扩展功能}
		\begin{itemize}[<+-| alert@+>]
		\item 支持步进、步退、自动运行、暂停等操作,并能以不同频率(最高1000Hz)运行。
		\item 显示内存中的数据并指示当前栈的位置,也可以监视一个指定的内存地址。
		\item 实现了{\bf Y86汇编器和反汇编器},从而能够接受.ys文件作为输入,也能保存汇编和反汇编的结果。
		\item 内存的实现采用了{\bf 分页技术}。
		\item 模拟实现了一个{\bf L1缓存的简化版}。
		\item 程序运行结束后可以生成{\bf 性能分析},包括对缓存命中和CPI的统计。
		\item 添加了一条新指令iaddl,并{\bf 归纳了添加指令的步骤}。
		\end{itemize}
	}
	
	\section{实现细节}
	
	\subsection{内核}
	
	\frame{
		\frametitle{内核}
		\begin{itemize}[<+-| alert@+>]
		\item 封装ALU、CPU,提供清晰的接口
		\item 数值传递、检查条件、读写内存
		\item 对照HCL实现流水线控制逻辑
		\item 编写测试用例来检查正确性
		\end{itemize}
	}
	
	\subsection{内存}
	
	\frame{
		\frametitle{内存}
		\begin{itemize}[<+-| alert@+>]
		\item Int8Array
		\item 使用分页技术
		\item 内存初始化为0xCC
		\item 多类型接口
		\end{itemize}
	}
	
	\subsection{缓存}
	
	\frame{
		\frametitle{缓存}
		\begin{itemize}[<+-| alert@+>]
		\item 模拟实现了一个{\bf L1 Cache}的简化版,$S = 2, E = 2, B = 64, m = 32$。
		\item {\bf 选取策略:} 由于数据的读写以$32$位为单位,所以两个相邻地址之间差为$4$,因此取$index$为地址的低第$4$位,$offset$为地址的低第$3$位,这样可以较好地利用{\bf Spatial Locality}。
		\item {\bf 抛弃策略:} 当两条Line的$Valid$位都为$1$时而又需要写入新数据时,选择访问次数少的一条Line写回内存。这样可以较好地利用{\bf Temporal Locality}。
		\item {\bf 写回策略:} 当一条Line的$Dirty$位为$1$,即数据被修改过时,才将其写回内存。
		\end{itemize}
	}
	
	\subsection{汇编\&反汇编}
	
	\frame{
		\frametitle{汇编\&反汇编}
		\begin{itemize}[<+-| alert@+>]
		\item 正则、转换、拼接
		\item 良好的封装、低耦合
		\item 方便添加新指令
		\end{itemize}
	}
	
	\section{测试用例}
	
	\frame{
		\frametitle{测试用例}
		\begin{table}[h]
		\begin{tabular}{|l|l|}
		\hline
		{\bf \#} & {\bf 描述}										\\ \hline
		1	& 测试样例,包括.long、循环和调用						\\ \hline
		2	& 测试模拟器是否能能正确结束运行						\\ \hline
		3	& 测试模拟器是否能处理Forward优先级					\\ \hline
		4	& 测试模拟器是否能处理Load/Use Hazard					\\ \hline
		5	& 测试模拟器是否能处理Combination A						\\ \hline
		6	& 对链表中的三个元素进行求和,测试内存取值相关操作		\\ \hline
		7	& 同上,但用递归完成,测试调用和返回等操作				\\ \hline
		\end{tabular}
		\end{table}
	}
	
\end{document}